\section{Method}

Let $P_\tau$ denote the observed transaction price at tick time $\tau$. The observed price can be decomposed as shown in Equation \ref{eq:price_decomp}:
\begin{equation}
    P_\tau = P_\tau^*+\eta_\tau
    \label{eq:price_decomp}
\end{equation}
where $P_\tau^*$ is the latent efficient price and $\eta_\tau$ represents microstructure noise. The pre-averaged price is defined in Equation \ref{eq:pre_avg} as
\begin{equation}
    \bar P_t = \frac{1}{k} \sum_{j=0}^{k-1} P_{t-j}
    \label{eq:pre_avg}
\end{equation}
where $t$ now indexes the last tick in each averaging block. Returns are then constructed as shown in Equation \ref{eq:returns}:
\begin{equation}
    r_t=\bar P_t - \bar P_{t-1}
    \label{eq:returns}
\end{equation}

\subsection{Baseline Auto-regressive Model}

As a benchmark, we consider a single-regime auto-regressive model of order one, given by Equation \ref{eq:baseline_ar}:
\begin{equation}
    r_t=\mu+\phi r_{t-1} + \varepsilon_t
    \label{eq:baseline_ar}
\end{equation}
where $\mu$ is the unconditional mean return, $\phi$ measures linear serial dependence and $\varepsilon_t$ is an innovative term with $\varepsilon_t\sim \mathcal{N}(0,\sigma^2)$.

\subsection{Hidden Markov Model with Regime-Switching Auto-regressive Dynamics}

Let $S_t \in \{ 1,\dots,K \}$ denote an unobserved discrete-time Markov chain representing the dependence regime at time $t$. The transition probabilities are defined in Equation \ref{eq:trans_probs}:
\begin{equation}
    \mathbb{P}(S_t=j \mid S_{t-1} = i) = p_{ij}
    \label{eq:trans_probs}
\end{equation}
with $p_{ij} \geq 0$ and $\sum_{j=1}^K p_{ij} = 1$ for all $i$.

Conditional on the regime $S_t=k$, returns follow an auto-regressive process described by Equation \ref{eq:regime_ar}:
\begin{equation}
    r_t = \mu_k + \phi_kr_{t-1} + \varepsilon_t, \quad \varepsilon_t \sim \mathcal{N}(0,\sigma_k^2)
    \label{eq:regime_ar}
\end{equation}

Here $\mu_k$ is the regime-specific mean, $\phi_k$ captures the strength of the short-term dependence in regime $k$ and $\sigma_k^2$ is the regime-specific innovation variance.

Regimes with $\phi_k$ close to zero are interpreted as noise-dominated periods, while regimes with larger positive $\phi_k$ correspond to short-term momentum or price returns.

\subsection{Modeling volatility}

Allowing $\sigma_k^2$ to vary across regimes captures abrupt changes in market uncertainty. As an extension, conditional heteroskedasticity can be introduced via a regime-dependent ARCH specification, shown in Equation \ref{eq:arch}:
\begin{equation}
    \sigma_{k,t}^2 = \omega_k + \alpha_k\varepsilon_{t-1}^2
    \label{eq:arch}
\end{equation}
where $\omega_k\geq0$ and $\alpha_k\geq0$. Building on this, volatility persistence can be incorporated through a regime-dependent GARCH specification, see Equation \ref{eq:garch}:
\begin{equation}
    \sigma_{k,t}^2 = \omega_k + \alpha_k\varepsilon_{t-1}^2 + \beta_k\sigma_{k,t-1}^2,
    \label{eq:garch}
\end{equation}
where $\beta_k\geq0$. This extension allows volatility shocks to decay gradually over time within each regime rather than affecting only the next period.

\subsection{Data}

Data was retrieved from \textcite{histdata}.

\subsection{Financial metrics}

\begin{equation}
    S_\text{avg} = \frac{1}{N}\sum_{i=1}^{N}\mid P_{\text{exec},i} - P_{\text{int},i}  \mid 
\end{equation}

\begin{equation}
    \text{Fill Rate} = \frac{\sum_{i=1}^{N} V_{\text{exec},i}}{\sum_{i=1}^{N}V_{\text{req},i}}
\end{equation}

\begin{equation}
    \overline{T} = \frac{1}{N}\sum_{i=1}^{N}\left( t_{\text{close},i} - t_\text{open},i \right)
\end{equation}

\begin{equation}
    MDD = \max_{\tau \in (0, T)} \left( \max_{t \in (0, \tau)} \frac{V(t) - V(\tau)}{V(t)} \right)
\end{equation}

\begin{equation}
    \text{Sharpe} = \frac{E[R_p - R_f]}{\sigma_p}
\end{equation}

\begin{equation}
    \text{Sortino} = \frac{E[R_p - R_f]}{\sigma_d}
\end{equation}

\begin{equation}
    W = \frac{N_{\text{win}}}{N_{\text{total}}}
\end{equation}

\begin{equation}
    PF = \frac{\sum \text{Gross Profits}}{\sum |\text{Gross Losses}|}
\end{equation}

\begin{equation}
    \text{PnL}_{\text{avg}} = \frac{1}{N} \sum_{i=1}^{N} \left( (P_{\text{exit}, i} - P_{\text{entry}, i}) \cdot Q_i - c_i \right)
\end{equation}

