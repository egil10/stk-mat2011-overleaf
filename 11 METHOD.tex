\section{Method}

Let $P_\tau$ denote the observed transaction price at tick time $\tau$. The observed price can be decomposed as 
\[
P_\tau = P_\tau^*+\eta_\tau
\]
where $P_\tau$ is the latent efficient price and $\eta_\tau$ represents microstructure noise. The pre-averaged price is defined as 
\[
\bar P_t = \frac{1}{k} \sum_{j=0}^{k-1} P_{t-j}
\]
where $t$ now indexes the last tick in each averaging block. Returns are then constructed as 
\[
r_t=\bar P_t - \bar P_{t-1}
\]

\subsection{Baseline Auto-regressive Model}

As a benchmark, we consider a single-regime auto-regressive model of order one
\[
r_t=\mu+\phi r_t{t-1} + \varepsilon_t
\]
where $\mu$ is the unconditional mean return, $\phi$ measures linear serial dependence and $\varepsilon_t$ is an innovative term with $\varepsilon_t\sim \mathcal{N}(0,\sigma^2)$. 

\subsection{Hidden Markov Model with Regime-Switching Auto-regressive Dynamics}

Let $S_t \in \{ 1,\dots,K \}$ denote an unobserved discrete-time Markov chain representing the dependence regime at time $t$. The transition probabilities are defined as 
\[
\mathbb{P}(S_t=j \mid S_{t-1} = i) = p_{ij}
\]
with $p_{ij} \geq 0$ and $\sum_{j=1}^K p_{ij} = 1$ for all $i$. 

Conditional on the regime $S_t=k$, returns follow an auto-regressive process
\[
r_t = \mu_k + \phi_kr_{t-1} + \varepsilon_t, \quad \varepsilon_t \sim \mathcal{N}(0,\sigma_k^2)
\]

Here $\mu_k$ is the regime-specific mean, $\phi_k$ captures the strength of the short-term dependence in regime $k$ and $\sigma_k^2$ is the regime-specific innovation variance. 

Regimes with $\phi_k$ close to zero are interpreted as noise-dominated periods, while regimes with larger positive $\phi_k$ correspond to short-term momentum or price returns. 

\subsection{Modeling volatility}

Allowing $\sigma_k^2$ to vary across regimes captures abrupt changes in market uncertainty. As an extension, conditional heteroskedasticity can be introduced via a regime-dependent ARCH specification
\[
\sigma_{k,t}^2 = \omega_k + \alpha_k\varepsilon_{t-1}^2
\]
where $\omega_k\geq0$ and $\alpha_k\geq0$. Building on this, volatility persistence can be incorporated through a regime-dependent GARCH specification,
\[
\sigma_{k,t}^2 = \omega_k + \alpha_k\varepsilon_{t-1}^2 + \beta_k\sigma_{k,t-1}^2,
\]
where $\beta_k\geq0$. This extension allows volatility shocks to decay gradually over time within each regime rather than affecting only the next period.

\subsection{Data}

Data was retrieved from \textcite{histdata}. 